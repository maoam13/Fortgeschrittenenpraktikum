\documentclass[12pt,a4paper]{article}
\usepackage[utf8]{inputenc}
\usepackage[german]{babel}
\usepackage[T1]{fontenc}
\usepackage{amsmath}
\usepackage{amsfonts}
\usepackage{amssymb}
\usepackage{graphicx}
\usepackage{siunitx}
\usepackage{float}
\usepackage[left=2cm,right=2cm,top=2cm,bottom=2cm]{geometry}
\author{Gerald}

\begin{document}
\sisetup{separate-uncertainty = true}
	\setlength{\parindent}{0pt} 
	\begin{center}
		{\LARGE Versuchsprotokoll}\\
		\begin{large}
			zum Fortgeschrittenenpraktikum im Bachelorstudiengang Physik\\[0.4cm]
			an der RWTH Aachen\\
			II. Physikalisches Institut A\\[5.5cm]
			\Large\textbf{\textsl{Nuclear Magnetic Resonance (NMR)}}\\[5.5cm]
			\normalsize\textit{vorgelegt\\von}\\[0.4cm]
			\large{Moritz Berger\\Gerald Kolter}\\[2cm]
			\large \textbf{Wintersemester 2017/18}
		\end{large}
	\end{center}
	\newpage
	
	\tableofcontents
	\newpage

\section{Ziel des Versuchs}
Das Ziel des Versuches besteht darin, die Relaxationszeiten von Wasserstoffkernen, also von Protonen, zu bestimmen. Diese geben an, wie schnell sich die Spins nach Auslenkung wieder in Richtung eines konstanten Magnetfeldes ausrichten.\\
Wenn sich Spins in einem konstanten Magnetfeld (im Folgenden mit $B_0$ bezeichnet) befinden, richten sie sich entlang dessen aus. Durch Anlegen eines Wechselfeldes (im Folgenden mit $B_1$ bezeichnet) senkrecht dazu, wird das effektive Magnetfeld zur Vektoraddition zwischen dem konstanten Magnetfeld und dem Wechselfeld. Die Spins präzedieren dann um das effektive Magnetfeld. Wird das Wechselfeld nur kurz angelegt, drehen die Spins keine ganze Rotation um das effektive Magnetfeld. Dadurch kann die Ausrichtung der Spins manipuliert werden. Im Folgenden sind sogenannte $\pi /2$-Pulse und $\pi$-Pulse relevant. Diese bezeichnen Pulse des Wechselfeldes, die die Spins um einen Winkel von $\pi /2$ bzw. $\pi$ drehen.\\
Bei der Wahl des Koordinatensystems wird die Z-Achse in Richtung des konstanten Magnetfeldes $B_0$ gelegt. Für die Betrachtung der Spinausrichtung wird in ein sich mit dem Spin um die z-Achse rotierendes Koordinatensystem gewechselt, sodass der Spin ein konstanter Vektor ist. Wenn die Frequenz der Pulse der Frequenz, mit der die Spins rotieren, der sogenannten Larmorfrequenz $\omega _L = \gamma \cdot B_0$ mit dem gyromagnetischen Verhältnis $\gamma$ entspricht, bewirkt ein $\pi /2$-Puls eine Drehung des Spins in die x-y-Ebene. Ein $\pi$-Puls spiegelt den Spin an der x-y-Ebene.


\section{Aufbau}
In ein konstantes Magnetfeld eines Permanentmagneten wird eine Spule so installiert, dass die Magnetfeldrichtung eines durch Induktion in dieser Spule erzeugten Magnetfeldes senkrecht auf dem Feld des Permanentmagneten steht. In diese Spule wird die Probe eingebracht. \\
Die Spule dient gleichzeitig als Sender für ein Wechselfeld und als Messinstrument für die Magnetisierung in der Richtung der Spule, da diese in der Spule einen Strom induziert.\\
Mit einem Frequenzgenerator wird ein elektrisches Wechselfeld erzeugt, das wiederum ein Magnetfeld in der Spule induziert. Da zeitlich kurze Magnetfelder benötigt werden, wird der Frequenzgenerator an einen Pulsgenerator angeschlossen. Der dadurch generierte Wechselfeldpuls wird auf die Spule gegeben. Der Pulsgenerator dient gleichzeitig als Triggersignal für das Oszilloskop.\\
Das elektrische Signal, das in der Spule durch die Magnetisierung induziert wird, wird in einem sogenannten Mixer mit dem Signal des Frequenzgenerators multipliziert. Dieses Signal wird auf dem Oszilloskop zusammen mit dem direkt aus der Spule entnommenen Signal angezeigt.\\
Der Verstärker bietet die Möglichkeit, die Verstärkung für einen kurzen Zeitraum auf Null zu stellen, das sogenannte blanking. Damit können die Pulse aus dem Oszilloskopbild rausgehalten werden.



\section{Durchführung}

\subsection{Vorversuche}
Vor der eigentlichen Messung müssen die Versuchsparameter richtig gewählt werden. Dazu wird anstelle der Probe eine kleine Spule, die anstelle der Empfängerspule an das Oszilloskop angeschlossen wird, in den Probenraum eingebracht. Dann werden die Frequenz ($\omega _p$) und die Höhe der Leiterschleife variiert, bis der induzierte Strom in der Leiterschleife maximal ist. Bei bekannten Größen der Leiterschleife (Querschnittsfläche $A$ und Anzahl Windungen $N$) kann daraus die Stärke des Wechselfeldes $B_1$ bestimmt werden:
\begin{equation*}
B_1 = \dfrac{U_0}{2 \omega _p N A}
\end{equation*}
Wenn die Frequenz $\omega _p \approx \omega _L$ gefunden ist, die Länge des $\pi /2$-Pulses bestimmt werden:
\begin{equation}
\label{T_pi/2}
T_{\pi /2} = \dfrac{\pi}{2} \cdot \dfrac{1}{\omega _p} = \dfrac{\pi}{2 \omega _p} = \dfrac{\pi}{2 \gamma B_1} = \dfrac{\pi N A \omega _p}{\gamma U_0}
\end{equation}
Wobei das gyromagnetische Verhältnis gegeben ist zu:
\begin{equation*}
\gamma _{Proton} = 2,675 \cdot 10^{8} \dfrac{1}{T s} 
\end{equation*}
Der $\pi$-Puls ist entsprechend doppelt so lang wie der $\pi /2$-Puls.\\
Mit diesen Voreinstellungen können nun die korrekten Parameter gewählt werden. Dazu wird die Probe in den Probenraum eingebracht. Der Einfachheit halber auf der Höhe, auf der vorher die Spule war. In dem direkten Signal aus dem Probenraum sollte nun gemäß Additionstheorem eine Schwebung
\begin{equation*}
2 \cdot \cos (\omega _L t) \cdot \cos (\omega _p t) = \cos ((\omega _L + \omega _p) t) \cdot \cos ((\omega _L - \omega _p) t)
\end{equation*}
zu sehen sein, wobei nur der Anteil mit der geringeren Frequenz zu sehen ist ($\propto \cos ((\omega _L - \omega _p) t)$). Wenn in den Daten keine Schwingung mehr zu sehen ist, ist damit offensichtlich der Punkt $\omega _p = \omega _L$ gefunden. Um diesen Punkt zu finden, müssen die Höhe der Probe, die Frequenz und mit der Frequenz gemäß Gl. \ref{T_pi/2} auch die Pulsdauer variiert werden.\\
Die einzustellenden Parameter sind abhängig von der Frequenz, die wiederum abhängig von der Stärke des konstanten Magnetfeldes ist. Da der Permanentmagnet temperaturabhängig ist, muss die Feineinstellung der Parameter zwischen den Messreihen wiederholt werden.


\subsection{Messung von T1}

\begin{figure}
\centering
\includegraphics[scale=0.8]{Bilder/F0003TEK.PNG}
\caption{Ergebnis einer einzelnen Messung zur Bestimmung von T1. Die blaue Kurve zeigt das zu untersuchende Signal.}
\label{fig:MessungT1_Beispiel}
\end{figure}

Für die Messung von T1 werden die Spins mit einem $\pi$-Puls umgedreht. Nach einer Wartezeit $\tau$, werden die restlichen, noch in der umgedrehten Position befindlichen Spins mit einem $\pi /2$-Puls in die x-y-Ebene gedreht, um dort vermessen zu werden. Das Ergebnis einer solchen Messung ist beispielhaft in Abbildung \ref{fig:MessungT1_Beispiel} gezeigt. Der erste Peak ist der Rest des $\pi$-Pulses. Der zweite Peak ist der Free Induction Decay (FID) nach dem $\pi /2$-Puls. \\
Für die Bestimmung von T1 wird die Wartezeit zwischen den beiden Pulse durchgefahren und die relative Höhe des FID zur Höhe des $\pi$-Pulses bestimmt. Die Erwartung ist ein exponentieller Abfall:
\begin{equation}
\label{T1_Exponentialfunktion}
U_{rel} (t) := \dfrac{U_{FID} (t)}{U_0} = 1 - 2 e^{-\frac{t}{T_1}}
\end{equation}

\begin{tabular}{|c|c|c|}
\hline 
Pulshöhe [mV] & FID-Höhe [mV] & $\tau$ [s] \\ 
\hline 
448 & -396 & 0.918 \\ 
\hline 
452 & -380 & 1.800 \\ 
\hline 
452 & -356 & 2.700 \\ 
\hline 
472 & -328 & 3.618 \\ 
\hline 
472 & -296 & 4.554 \\ 
\hline 
452 & -268 & 5.436 \\ 
\hline 
460 & -236 & 6.300 \\ 
\hline 
448 & -204 & 7.290 \\ 
\hline 
460 & -176 & 8.154 \\ 
\hline 
456 & -136 & 9.036 \\ 
\hline 
456 & -124 & 9.954 \\ 
\hline 
452 & -92 & 10.854 \\ 
\hline 
444 & -72 & 11.862 \\ 
\hline 
456 & -52 & 12.978 \\ 
\hline 
448 & -40 & 13.428 \\ 
\hline 
440 & -36 & 14.040 \\ 
\hline 
444 & -32 & 14.148 \\ 
\hline 
452 & -28 & 14.490 \\ 
\hline 
448 & 28 & 14.850 \\ 
\hline 
448 & 44 & 15.300 \\ 
\hline 
440 & 40 & 15.516 \\ 
\hline 
440 & 52 & 15.858 \\ 
\hline 
448 & 60 & 16.326 \\ 
\hline 
440 & 56 & 17.118 \\ 
\hline 
448 & 80 & 18.000 \\ 
\hline 
444 & 108 & 18.864 \\ 
\hline 
444 & 128 & 19.818 \\ 
\hline 
452 & 136 & 20.718 \\ 
\hline 
444 & 160 & 21.636 \\ 
\hline 
452 & 176 & 22.500 \\ 
\hline 
440 & 196 & 23.400 \\ 
\hline 
432 & 200 & 24.336 \\ 
\hline 
444 & 212 & 25.218 \\ 
\hline 
436 & 228 & 26.100 \\ 
\hline 
444 & 248 & 27.090 \\ 
\hline 
416 & 252 & 27.990 \\ 
\hline 
420 & 260 & 28.800 \\ 
\hline 
424 & 272 & 29.700 \\ 
\hline 
432 & 296 & 30.600 \\ 
\hline 
404 & 300 & 30.690 \\ 
\hline 
416 & 292 & 31.500 \\ 
\hline 
412 & 304 & • \\ 
\hline 
412 & 364 & • \\ 
\hline 
\end{tabular} 



\subsection{Messung von $T_2$}

Für die Messung von $T_2$ werden die Spins mit einem  $\dfrac{\pi}{2}$ Puls in die x-y-Ebene gedreht. Durch Inhomogenitäten im Magnetfeld präzedieren die Spins allerdings unterschiedlich schnell, wodurch der Magnetisierungsvektor verschmiert. Deswegen wird nach einer Wartezeit $\tau$ ein $\pi$ Puls angelegt, der die Spins innerhalb der Ebene spiegelt und somit die Verschmierungsrichtungs umdreht. Dadurch entsteht nach einer Zeit von insgesamt $2 \tau$ ein Spinecho. \\ 
\\
Um außerdem möglichen Diffusionsprozessen entgegen zu wirken gibt es nun 2 unterschiedliche Verfahren:

\subparagraph{Carr-Purcell-Pulssequenz}
Bei diesem Verfahren wird nach dem anfänglichen $\dfrac{\pi}{2}$ und $\pi$ Pulsen alle  $2 \tau$ ein neuer $\pi$ Puls angelegt. Dadurch erhält man mehrere Spinechos bei einer Pulssequenz. Durch die wesentlich schnellere Durchführung werden so Diffusionseffekte minimiert.
\subparagraph{Meiboom-Gill-Pulssequenz}
Dieses Verfahren besteht aus exakt derselben Pulssequenz, wie das Carr-Purcell-Verfahren. Der große Unterschied ist aber, dass  die $\pi$ Pulse um $90^{\circ}$ zum anfänglichen $\dfrac{\pi}{2}$ Puls verschoben ist. Wurde der $\pi$ Puls beim C-P-Verfahren also in x-Richtung (im ruhendem System) angelegt, so befindet er sich jetzt in y-Richtung.\\
(EDIT)Falls nun der $\pi$ Puls in exakt um $180^{\circ}$ dreht, sondern etwas weniger oder mehr, dann addieren sich diese Verschiebungen beim C-P-Verfahren immer weiter auf. Dadurch weicht der Spinvektor immer weiter von der x-y-Ebene ab und man bekommt ein zu kleines Ergebnis für die Relaxationszeit.\\
Beim M-G-Verfahren finden die Verschiebungen nach jedem Puls in die entgegengesetzt Richtung statt. Somit bleit der Spinvektor in der Nähe der x-yEbene und der Abweichungseffekt wird somit negiert.\\
\\
Es wurden zu beiden Methoden Datensätze für verschiedene $\tau$ aufgezeichnet, um bei der Auswertung entscheiden zu können, welche Methode die besseren Ergebnisse liefert.\\
Dabei wird das $\tau$ möglichst klein gewählt, allerdings groß genug, damit die Peaks nicht untereinander Interferieren und noch gut erkennbar sind. Im vorliegendem Aufbau ergab ein $\tau$ zwischen 4-7ms ein gutes Bild.\\

\begin{figure}
\centering
\includegraphics[scale=0.8]{Bilder/T2Beispiel.png}
\caption{Datensatz zur Bestimmung von $T_2$ beispielhaft für die Meiboom-Gill-Sequenz mit $\tau = 6ms$. Es wurde gekennzeichnet, um welche Art von Peak es sich handelt. }
\label{fig:T2Beispiel}
\end{figure}

\section{Ergebnisse}

\subsection{T1}
\subsubsection{Zero-crossing-point-Methode}
\subsubsection{Linearer Fit an halblogarithmischer Auftragung}
\subsubsection{Exponentialfit}



\subsection{T2}
\subsubsection{Messdaten}\label{sec:T2messdaten}
\begin{figure}
\centering
\includegraphics[scale=0.8]{Bilder/T2CP.png}
\includegraphics[scale=0.8]{Bilder/T2MG.png}
\caption{Ergebnisse der Pulssequenzen zur Bestimmung von $T_2$ mit $\tau = 4ms$ und verstelltem Pulslängen. \textbf{Oben}: Carr-Purcell-Verfahren. \textbf{Unten}: Meiboom-Gill-Verfahren. Auffällig sind die sichtbaren $\pi$ Pulse und das Fehlen des 3.Spinechos beim C-P-Verfahren.}
\label{fig:T2Daten}
\end{figure}

Die ersten Messungen für $T_2$ sind in Abbildung \ref{fig:T2Daten} für beide Verfahren beispielhaft dargestellt.\\
Bei dieser Messreihe ist auffällig, dass die $\pi/2$ Pulse und deren FID sichtbar sind, was bei idealen Einstellungen und Spinverhalten nicht der Fall sein sollte. Außeredem verschwindet bei der C-P-Methode das 3.Spinecho.\\
Das Auftreten der $\pi/2$ Pulse kann verschiedene Ursachen haben. Einerseits kann die RF-Kreisfrequenz falsch eingestellt sein. Dadurch befindet sich das Magnetfeld bei den Pulsen nicht in der x-y Ebene, wodurch die Spinausrichtung nach einem $\pi$ Puls eine horizontale Komponente hat, welche natürlich gemessen weir.\\
Es kann außerdem sein, dass die Pulslängen falsch eingestellt waren. Durch eine Kombination mit der falschen Frequenz kann dies dazu führen, dass der $\pi$ Puls die in der Druchführung beschriebe Verschmierung teilweise aufgehebt, wodurch ein messbarer Peak entsteht.\\
Falsch eingestellte Pulslängen würden auch das Fehlen des 3.Spinechos  bei der C-P-Methode erklären, da sich die Abweichungen von der x-y Ebene hier wie oben beschrieben aufaddieren und der Spinvektor somit nach mehreren $\pi$ Pulsen im Extremfall in z-Richtung zeigt und somit nicht gemessen werden kann.\\
\\
Um diese Effekte zu minimieren wurde eine zweite Messreihe aufgenommen, bei der die Frequenz nochmals neu kalibiert wurde und die Pulslängen so angepasst wurden, dass die $\pi$ Pulse möglichst klein ausfallen. Diese Messreihe ist beispielhaft in Abblidung \ref{fig:T2Datenalt} zu sehen.\\
Es ist dort wie erwartet zu sehen, dass die Peaks bei der C-P-Methode wegen der Abweichung von der x-y Ebene schneller abfallen. Dies ist ein unerwünschter Effekt, der das Ergebnis unter Umständen stark beeinflusst. Außerdem sind die Peaks ab einer gewissen Ordnung sehr klein und sogar teilweise gar nicht mehr sichtbar, was das Ergebnis deutlich ungenauer machen würde.\\
Deswegen wurde entschieden die Daten des Meiboom-Gill-Verfahrens für die Auswertung zu benuten.
\begin{figure}
\centering
\includegraphics[scale=0.8]{Bilder/T2CPalt.png}
\includegraphics[scale=0.8]{Bilder/T2MGalt.png}
\caption{Ergebnisse für $T_2$ mit angepasster Frequenz und Pulslänge für $\tau = 4ms$. \textbf{Oben}: Carr-Purcell-Verfahren. \textbf{Unten}: Meiboom-Gill-Verfahren. Für die Auswertung relevant ist nur die "Blaue" Kurve von Channel 2, welche die Amplitude des Messsignals beschreibt.}
\label{fig:T2Datenalt}
\end{figure}

\subsubsection{Peakbestimmung}
Mithilfe eines Programmes wurden die vom Oszilloskop gegebenen Datensätze eingelesen.
Es wurde wie bei der Bestimmung von $T_1$ wieder eine Rauschmessung durchgeführt.
Diese ergab einen Offset von $U_{Offset} = 20.6mV$ und eine Schwankung von $\sigma = 4mV$.\\
Die Peakhöhe und Lage wurde durch eine lokale Maximumsbestimmung in der nähe der Peaks durchgeführt.\\
Dabei wurde der Peak des $\pi/2$ Pulses ausgenommen, da dieser nicht durch ein Spinecho entsteht und der Puls somit das Ergebnis verfälschen könnte.\\
Bei allen Datensätzen wurde eine Zeitskala von 10ms und eine Höhenskala von 200mV gewählt. Bei diesen Skalen bertägt die Digitalisierung 8mV und 0.04ms.\\
Da das Maximum oft über mehrere Digitalisierungsschritte verteilt war, wurde angenommen, dass das wahre Maximum gleichverteilt zwischen diesen Schritten liegt. Eine gute Wahl der Maximumsbreite betrug $\Delta t = 0.16ms$.\\
Damit ergibt sich eine Ableseungenauigkeit auf die Lage von

\begin{equation}
\sigma_t = \dfrac{Breite}{\sqrt{12}} \approx 0.05ms
\end{equation}

Der Fehler auf die Peakhöhe ergibt sich durch die Schwankung der Werte innerhalb der Maximumsbreite. In diesem Fehler steckt sowohl der Digitalisierungsfehler, als auch der Fehler durch das Rauschen, weswegen diese Fehler nicht mehr extra einbezogen werden müssen.

\begin{table}
\begin{tabular}{|c|c|}
\hline 
Position[ms] & Höhe[mV]\\ 
\hline
$ 8.17 \pm 0.05 $ & $ 437.01 \pm 11.31 $ \\
\hline
$ 16.08 \pm 0.05 $ & $ 309.01 \pm 6.53 $ \\
\hline
$ 24.06 \pm 0.05 $ & $ 213.01 \pm 6.53 $ \\
\hline
$ 32.04 \pm 0.05 $ & $ 167.01 \pm 3.77 $ \\
\hline
$ 40.03 \pm 0.05 $ & $ 123.01 \pm 7.54 $ \\
\hline
$ 48.03 \pm 0.05 $ & $ 103.01 \pm 3.77 $ \\
\hline
$ 56.02 \pm 0.05 $ & $ 71.01 \pm 3.77 $ \\
\hline
$ 64.02 \pm 0.05 $ & $ 67.01 \pm 7.54 $ \\
\hline
$ 72.02 \pm 0.05 $ & $ 47.01 \pm 3.77 $ \\
\hline
$ 80.02 \pm 0.05 $ & $ 39.01 \pm 3.77 $ \\
\hline
\end{tabular}
\caption{Peaks für $\tau = 4ms$ }
\end{table}

\subsubsection{Exponentialfit}

\begin{figure}
\centering
\includegraphics[scale=0.9]{Bilder/T2exp.png}
\caption{Exponentialfit}
\label{fig:T2exp}
\end{figure}

An die gefundenen Peaks wird nun eine Exponentialfunktion der Form

\begin{equation}
y = b \cdot e^{-a\cdot t}
\end{equation}
gefittet. Dies ist in Abbildung \ref{fig:T2exp} dargestellt.\\
Aus den gefundenen Werten berechnet sich $T_2$ durch 
\begin{equation}
T_2 = \dfrac{1}{a} = 28.1ms
\end{equation}
und der Fehler durch
\begin{equation}
\sigma_{T_2} = \dfrac{1}{a^2} \cdot \sigma_a = 1.1ms
\end{equation}

\subsubsection{Linearer Fit an Halblogarythmischer Auftragung}

\begin{figure}
\centering
\includegraphics[scale=0.9]{Bilder/T2lin.png}
\caption{Linearer Fit}
\label{fig:T2lin}
\end{figure}

Nun werden die Spannungswerte logarythmisiert und eine lineare Funktion an die Daten gefittet:
\begin{equation}
y = a \cdot t + b
\end{equation}
Daraus erhält man $T_2$ wieder durch 
\begin{equation}
T_2 = \dfrac{-1}{a} = 28.3ms
\end{equation}
und der Fehler durch
\begin{equation}
\sigma_{T_2} = \dfrac{1}{a^2} \cdot \sigma_a = 1.0ms
\end{equation}

\subsubsection{Weitere Auswertungen}

Um mögliche Systematiken abschätzen zu können wurden mit den oben beschriebenen Methoden weitere Datensätze ausgewertet. Alle Ergebnisse sind in Tabelle \ref{T2zusammenfassung} aufgelistet.\\
Zuerst wurde die Messung mit dem Beiboom-Gill Verfahren für verschiedene $\tau$ wiederholt, um ein genaueres Resultat zu bekommen und mögliche statistische Schwankungen und Fehlmessungen zwischen verschiedenen Messungen zu eliminieren.\\
Die Ergenisse stimmen sowohl für verschiedene Messreihen als auch für die beiden verschiedenen Auswertungsmethoden innerhalb ihrer Fehler überein. 	Eine gewichtete Mittelung über die vier Messungen liefert:

\begin{equation}
\boxed{T_2(Exponential) = (27.6\pm 0.5)ms}
\end{equation}
\begin{equation}
\boxed{T_2(Linear) = (27.8\pm 0.5)ms}
\end{equation}\\

Anhand der Messreihe, bei der Frequenz und Pulslängen verstellt waren, kann außerdem der Einfluss von verstellten Parametern beobachtet werden. Es ist auffällig, dass die Ergebnisse der einzelnen Messungen sehr viel stärker schwanken und im Mittel mit $T_2 = (28.9\pm0.8)ms$ größer sind. Dies war zu erwarten, da sich die Magnetisierung hier durch die in Kapitel \ref{sec:T2messdaten} beschriebenen Effekte unkontrollierter verhält.\\
\\
Zuletzt wurde noch das Carr-Purcell-Verfahren ausgewertet. Es ist zu sehen, dass dieses Verfahren eine wesentlich kleineres $\tau$ liefert, da die Pulslängen nicht genau genug eingestellt werden konnten. Bei stark verstellten Parametern kann man die Daten sogar nicht mehr wirklich auswerten, was an dem deutlich größeren $\chi^2/ndof$ der alternativen C-P-Messung (vgl. Tabelle \ref{T2zusammenfassung}) zu sehen ist.

\begin{table}
\begin{tabular}{|c|c|c||c|c||c|c|}
\hline
Verfahren & $\tau [ms]$ & Peaks &  $T_2$ (Exponentialfit) & $\chi^2/ndof$ &$T_2$ (Linerarer Fit) & $\chi^2/ndof$\\
\hline
M-G & $4$ & 10 & $ 28.1 \pm 1.1 $ & $ 3.71 $ & $ 28.3 \pm 1.0 $ & $ 3.11 $ \\
\hline
M-G & $5$ & 9 & $ 27.4 \pm 1.0 $ & $ 2.73 $ & $ 27.7 \pm 1.0 $ & $ 2.48 $ \\
\hline
M-G & $6$ & 7 & $ 27.3 \pm 0.8 $ & $ 1.83 $ & $ 27.5 \pm 0.8 $ & $ 1.43 $ \\
\hline
M-G & $7$ & 6 & $ 27.7 \pm 1.3 $ & $ 2.53 $ & $ 27.9 \pm 1.3 $ & $ 1.83 $ \\
\hline
\hline
M-G(alt.) & $4$ & 10 & $ 27.3 \pm 1.0 $ & $ 2.4 $ & $ 27.4 \pm 0.9 $ & $ 1.89 $ \\
\hline
M-G(alt.) & $5$ & 9 & $ 30.7 \pm 1.1 $ & $ 2.49 $ & $ 31.0 \pm 1.1 $ & $ 1.8 $ \\
\hline
M-G(alt.) & $6$ & 7 & $ 28.6 \pm 1.3 $ & $ 1.71 $ & $ 28.8 \pm 1.2 $ & $ 1.15 $ \\
\hline
M-G(alt.) & $7$ & 6 & $ 29.5 \pm 1.9 $ & $ 4.95 $ & $ 29.8 \pm 2.0 $ & $ 3.22 $ \\
\hline
\hline
C-P & $4$ &  & $17.6\pm 0.4$ & $1.61$ & $17.7\pm 0.4$ & $1.02$\\
\hline
C-P(alt.) & $4$ &  & $ 8.2 \pm 1.1 $ & $ 21.13 $ & $ 11.2 \pm 2.1 $ & $ 31.42 $ \\
\hline
\end{tabular}
\caption{Übersicht aller ausgewerteten Ergebnisse. Dabei ist für jede Messung das verwendete Verfahren, die Wertezeit $\tau$ und die Anzahl der ausgewerteten Spinechos angegeben. Die Kennzeichnung (alt.) bedeutet, dass die Daten aus der alternativen Messung stammen, bei der Frequenz und Pulslänge falsch eingestellt waren.}
\label{T2zusammenfassung}
\end{table}


\end{document}