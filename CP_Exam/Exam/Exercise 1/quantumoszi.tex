\documentclass[12pt,a4paper]{article}
\usepackage[utf8]{inputenc}
\usepackage[english]{babel}
\usepackage[T1]{fontenc}
\usepackage{amsmath}
\usepackage{amsfonts}
\usepackage{amssymb}
\usepackage{graphicx}
\usepackage{verbatim}
\usepackage[left=2cm,right=2cm,top=2cm,bottom=2cm]{geometry}
\usepackage{siunitx}
\usepackage{placeins}
\author{Tim}
\usepackage{gensymb}
\usepackage{mathtools}
\usepackage{physics}
\begin{document}

\section{Introduction}

In this exercise the behavior of a quantum harmonic oscillator is analyzed. It is described by a wave function in a quadratic potential
$$
V = \dfrac{\Omega^2}{2} x^2
$$
where $\Omega$ denotes the frequency of the oscillator. The equation of motion of this system is given by the one dimensional Schrödinger equation:
\begin{equation}
i \dfrac{\partial}{\partial t} \Phi(x,t) = \left(-\dfrac{1}{2}\dfrac{\partial^2}{\partial x^2} + \dfrac{\Omega^2}{2} x^2\right) \Phi(x,t) = H \Phi(x,t)
\end{equation}
with all variables being dimensionless ($\hbar = m = 1$).\\
The goal is to simulate the time evolution of the wave function for different starting functions and to compute the average and variance of this function over time.
\section{Simulation Model and Method}

The simulated area is a one dimensional grid ranging from $-15\leq x\leq 15$ that is split into $L=1001$ equally spaced points, which results in a step width of $\Delta = 0.03$. The initial wave function is described by a Gaussian function:
\begin{equation}
\Phi_0(x,t=0) = \dfrac{1}{\pi \sigma^2} e^{-(x-x_0)^2/2\sigma^2}
\end{equation}
6 different combinations of the oscillator frequency $\Omega$, $\sigma$ and $x_0$ are simulated. 

The Schrödinger equation is solved by using a second order product formula approach.

\end{document}