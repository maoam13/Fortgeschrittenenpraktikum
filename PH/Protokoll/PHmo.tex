\documentclass[12pt,a4paper]{article}
\usepackage[utf8]{inputenc}
\usepackage[german]{babel}
\usepackage[T1]{fontenc}
\usepackage{amsmath}
\usepackage{amsfonts}
\usepackage{amssymb}
\usepackage{graphicx}
\usepackage{siunitx}
\usepackage{float}
\usepackage[left=2cm,right=2cm,top=2cm,bottom=2cm]{geometry}
\usepackage{hyperref}
\author{Gerald}

\begin{document}
\sisetup{separate-uncertainty = true}
	\setlength{\parindent}{0pt} 
	\begin{center}
		{\LARGE Versuchsprotokoll}\\
		\begin{large}
			zum Fortgeschrittenenpraktikum im Bachelorstudiengang Physik\\[0.4cm]
			an der RWTH Aachen\\
			II. Physikalisches Institut A\\[5.5cm]
			\Large\textbf{\textsl{Magnetische Phasenübergänge (PH)}}\\[5.5cm]
			\normalsize\textit{vorgelegt\\von}\\[0.4cm]
			\large{Moritz Berger (355244)\\Gerald Kolter (355005)}\\\textbf{Gruppe 30}\\[2cm]
			\large \textbf{Wintersemester 2017/18}
		\end{large}
	\end{center}
	\newpage
	
	\tableofcontents
	\newpage

\section{Versuchsziel}
Ziel des Versuchs ist es die Sprungtemperatur eines Hochtemperatursupraleiters und die Curie-Temperatur und dadurch die Zusammensetzung einer GdAg$_{1-x}$Zn$_x$-Probe zu bestimmen.

\section{Aufbau}
Der Messaufbau für den Hauptversuch besteht aus einem Hartshorn-Spulensystem, das zusammen mit der darin befindlichen Probe in flüssigem Stickstoff abgekühlt wird. Mit einem Lockin-Verstärker wird die Differenz zwischen den Signalen der beiden gegenläufigen Empfängerspulen gemessen und über ein Multimeter mit dem Messrechner angeschlossen. Unter dem Hartshorn-Spulensystem befindet sich eine Si-Diode zur Messung der Temperatur, die ebenfalls an den Messrechner angeschlossen ist.

\section{Durchführung}
\subsection{Vorversuche}
\subsubsection{Untersuchung eines Tiefpasses}

\begin{figure}
\centering
\includegraphics[scale=0.1]{Bilder/Vorversuch1/Tiefpass_Schaltbild.png}
\caption[test]{Schaltbild\footnotemark eines Tiefpasses.}
\label{fig:Tiefpass_Schaltbild}
\end{figure}
\footnotetext{Quelle: http://de.wikipedia.org/wiki/Tiefpass}

Mit einem $\SI{1,5}{k \Omega}$ Widerstand und einem $\SI{100}{nF}$ Kondensator wird ein Tiefpass zusammengesetzt. Mit einem Frequenzgenerator wird eine sinusförmige Wechselspannung an den Tiefpass angelegt. Abbildung \ref{fig:Tiefpass_Schaltbild} zeigt das entsprechende Schaltbild. Die angelegte Frequenz wird durchgefahren und dabei werden die an den Tiefpass angelegte und die durch den Tiefpass gefilterte Wechselspannung auf dem Oszilloskop gemessen.

\subsubsection{Untersuchung der Filter des Lockin-Verstärkers}
Das Referenzsignal des Lockin-Verstärkers wird auf einen Eingang des Lockin-Verstärkers gelegt. Die Filterfrequenz wird auf $f_0 = \SI{1000}{Hz}$ eingestellt. Es werden alle drei Filter (Tiefpass-, Hochpass- und Bandpassfilter) des Lockin-Verstärkers vermessen, indem jeweils die Frequenz des Referenzsignals durchgefahren und die Spannungsausgabe des Lockin-Verstärkers gemessen wird.

\subsubsection{Signalfiltern mit Tiefpass und Hochpass}
Mit einem Frequenzgenerator wird eine niedrige Frequenz erzeugt und an einen Eingang des Lockin-Verstärkers gelegt. An den anderen Eingang des Lockin-Verstärkers wird das Referenzsignal des Lockin-Verstärkers gelegt. Mit dem Oszilloskop werden sowohl das Signal des Frequenzgenerators, das Referenzsignal und die Überlagerung der beiden Signale, die der Lockin-Verstärker auf dem SIG.MON Ausgang ausgibt.

\subsubsection{Zeitkonstante und Empfindlichkeit}

\begin{table}
\centering
\begin{tabular}{|c|c|}
\hline 
Integrationszeit $T_I$ & \SI{10}{ms} \\ 
\hline 
Amplitude $U_0$ & \SI{0,5}{V} \\
\hline 
Verstärkung $s$ & \SI{200}{mV} \\ 
\hline 
Phasenschieber & $\varphi = \frac{\pi}{2}$ \\ 
\hline 
\end{tabular} 
\caption{Einstellungen zur Bestimmung einer geeigneten Zeitkonstanten und Empfindlichkeit.}
\label{tab:Zeitkonst_Einstellungen}
\end{table}

Der Lockin-Verstärker integriert das Signal über einen einstellbaren Zeitraum, um so die Schwingung aufgrund der Orthogonalitätsbedingung aus dem Ausgangssignal zu integrieren. \\
Die Empfindlichkeit $s$ des Lockin-Verstärkers wird angegeben in der Spannung, die auf \SI{10}{V} verstärkt wird, sodass sich der Verstärkungsfaktor zu $v = \frac{\SI{10}{V}}{s}$ berechnet. \\
Das Referenzsignal wird auf den Eingang des Lockin-Verstärkers gelegt. Die Frequenz des Referenzsignals wird durchgefahren, um ein geeignetes Verhältnis zwischen Referenzfrequenz und Integrationszeit zu finden. Tabelle \ref{tab:Zeitkonst_Einstellungen} zeigt die verwendeten Einstellungen.

\subsection{Hauptversuch}
\subsubsection{Messung des Supraleiters}

\begin{table}
\centering
\begin{tabular}{|c|c|}
\hline 
Integrationszeit $T_I$ & \SI{10}{ms} \\ 
\hline 
Sensitivität $s$ & \SI{500}{\mu V} \\ 
\hline
Filter & Bandpass \\
\hline
Güte & 1 \\
\hline
Filterfrequenz & \SI{380}{Hz} \\
\hline 
Phasenschieber & $\varphi$ = 84.5$^{\circ}$ \\ 
\hline 
\end{tabular} 
\caption{Einstellungen für die Messung des Supraleiters.}
\label{tab:Supra_Einstellungen}
\end{table}

Tabelle \ref{tab:Zeitkonst_Einstellungen} zeigt die Einstellungen, die für die Messungen mit dem Supraleiter verwendet wurden. \\
Der Messstab mit dem Supraleiter wird zunächst mit einer Pumpe evakuiert und anschließend mit Helium als Kontaktgas befüllt. Das Ende des Stabes, an dem die Messeinrichtung sitzt, wird zum Kühlen in einen mit flüssigem Stickstoff gefüllten Dewar getaucht und auf ca. \SI{80}{K} gekühlt. Für die Messung wird anschließend der Messstab aus dem Stickstoff herausgezogen, allerdings nur so weit, dass die Erwärmungsrate $\frac{\Delta T}{\Delta t}$ unter \SI{0,1}{K/s} bleibt. Dies ist wichtig, um sicherzustellen, dass zu jedem Zeitpunkt eine möglichst homogene Temperaturverteilung gegeben ist. Während der Erwärmung von \SI{80}{K} auf ca. \SI{120}{K} wird die Temperatur und die Differenzspannung zwischen den beiden Empfängerspulen des Hartshorn-Spulensystems gemessen. \\
Diese Messung wird einmal ohne Probe als Untergrundmessung zur Korrektur und zweimal mit dem Supraleiter durchgeführt: Einmal mit einer Phasenverschiebung zwischen Eingangssignal und Referenzsignal des Lockin-Verstärkers von $\varphi = 0$ und einmal mit einer Phasenverschiebung von $\varphi = \frac{\pi}{2}$, sodass der Realteil ($\varphi = 0$) und der Imaginärteil ($\varphi = \frac{\pi}{2}$) der Suszeptibilität gemessen werden.

\subsubsection{Messung der Probe}
Bei der Vermessung der GdAg$_{1-x}$Zn$_x$-Probe wird nur der Imaginärteil gemessen. Die Messung wurde ebenfalls bei ca. \SI{80}{K} gestartet, jedoch bis ca. \SI{190}{K} aufgenommen. Aufgrund eines betragsgroßen Offset war der Betrag der an den Lockin-Verstärker angelegten Spannung zu groß, sodass hier die Sensitivität auf \SI{1}{mV} erhöht werden musste.

\section{Ergebnisse}
\subsection{Vorversuche}
\subsubsection{Untersuchung eines Tiefpasses}
\subsubsection{Untersuchung der Filter des Lockin-Verstärkers}
\subsubsection{Signalfiltern mit Tiefpass und Hochpass}
\subsubsection{Zeitkonstante und Empfindlichkeit}
\subsection{Hauptversuch}
\subsection{Messung des Supraleiters}
\subsubsection{Phasenkalibration}
\begin{figure}
\centering
\includegraphics[scale=0.5]{Bilder/Haupt_Supra/Kalialt.png}
\includegraphics[scale=0.5]{Bilder/Haupt_Supra/Kalialt_2.png}
\caption{Ursprüngliche Phasenkalibration, bei der die Spannung mithilfe der Phase auf 0 kalibriert wurde. Im rechten Bild wurde der Verlauf von $\chi'$ in grün und der von $\chi''$ in rot gekennzeichnet.}
\label{fig:Supra_Kalialt}
\end{figure}


Um Real- und Imaginärteil möglichst unabhängig voneinander betrachten zu können muss die Phase möglichst genau auf $90^\circ$ bzw. auf $0^\circ$ kalibriert werden. Ursprünglich sollte dies mithilfe der Tatsache geschehen, dass für den Supraleiter
\begin{equation}
\chi_C = -1-i\cdot 0
\end{equation}
gilt, der Imaginärteil also 0 ist.\\
Man verucht also auf $90^\circ$ so zu kalibrieren, dass der Offset \SI{0}{V} beträgt. Eine Messreihe mit einer solchen Kalibration ist in Abbildung \ref{fig:Supra_Kalialt} dargestellt.\\
Das Problem dabei ist, dass die Position des Dewar-Gefäßes das Magnetfeld der Spulen beeinflusst. Da diese Postition mehrfach während des Versuches geändert werden muss, entsteht ein Offset, der nicht durch das Potentiometer ausgeglichen wurde. Den Einfluss kann man zum Beispiel durch eine Änderung in der Spannung ganz am Anfang der Messung beobachten, wo das Spulensystem leicht aus dem Dewar angehoben wird, um den Erwärmungsvorgang zu starten.\\
Bei der 0-Kalibration kalibriert man somit nicht auf $90^\circ$, sondern auf irgendeinen anderen Wert. Dies zeigt sich in Abbildung \ref{fig:Supra_Kalialt} durch das gleichzeitige Auftreten der $\chi'$-Kurve (Änderung von $\chi' = -1$ zu $\chi' << 1$ an der kritischen Temperatur) und des $\chi''$-Peaks.\\
\\
Um die Phase besser kalibriern zu können wurde ein alternatives Vervahren verwendet. Es wurde zuerst für eine Phase von $90^\circ$ ein Kurvenverlauf aufgezeichnet und dann geschaut, wie sich dieser Verlauf ändert, wenn man die Phase leicht verringert. Dies wird solange wiederholt, bis der Unterschied vor und hinter der Sprungtemperatur, der durch die $\chi'$-Kurve entsteht, verschwindet. Dieses Verfahren ist in Abbildung \ref{fig:Supra_Kali} anhand von 4 Schritten dargestellt.

\begin{figure}
\centering
\includegraphics[scale=0.5]{Bilder/Haupt_Supra/Kal_2.png}
\includegraphics[scale=0.5]{Bilder/Haupt_Supra/Kal_0.png}
\includegraphics[scale=0.5]{Bilder/Haupt_Supra/Kal_1.png}
\includegraphics[scale=0.5]{Bilder/Haupt_Supra/Kal_3.png}
\caption{Alternative Phasenkalibration. \textbf{Oben links:} Aufwärmvorgang mit Phaseneinstellung von $90^\circ$. Der Verlauf der $\chi'$-Kurve ist sichtbar. \textbf{Oben rechts:} Aufwärmvorgang mit etwas besserer Phase von $86^\circ$. $\chi'$-Kurve ist schwächer ausgeprägt. \textbf{Unten links:} Abkühlvorgang mit zu kleiner Phase von $82^\circ$. $\chi'$-Kurve ist wieder stärker ausgeprägt. \textbf{Unten rechts:} Abkühlvorgang mit entgültiger  Phase von $82^\circ$. Die Spannung hat vor und hinter dem Peak den gleichen Wert.}
\label{fig:Supra_Kali}
\end{figure}

\subsubsection{Untergrundmessung}
\begin{figure}
\centering
\includegraphics[scale=0.8]{Bilder/Haupt_Supra/Untergrund.png}
\caption{Untergrundmessung mit leerem Probenbehälter.}
\label{fig:Supra_Untergrund}
\end{figure}

In Abbildung \ref{fig:Supra_Untergrund} ist die Untergundmessung dargestellt. Am Anfang der Messung (bei nirdrigen Temperaturen) sieht man einen plötzlichen Anstieg der gemessenen Spannung. Dieser entsteht weider dadurch, dass der Stab leicht aus dem Dewar-Gefäß angehoben wird. Aus diesem Grund können nur Messwerte ab c.a 81.5K benutzt werden, falls ein Untergrundabzug durchgeführt wird.\\
Oberhalb dieser Temperatur folgen die Daten keinem einfachen Modell. Da die genaue Ursache dieses Verhaltens unbekannt ist, konnte keine Anpassung an die Daten durchgeführt werden. Stattdessen wird der Untergrund direkt von den Messreihen abgezogen.

\subsubsection{Auswertung der $\chi'$-Messung}
\begin{figure}
\centering
\includegraphics[scale=0.8]{Bilder/Haupt_Supra/X1roh.png}
\caption{Rohdaten der $\chi'$-Messung}
\label{fig:Supra_X1roh}
\end{figure}

\begin{figure}
\centering
\includegraphics[scale=0.8]{Bilder/Haupt_Supra/X1_spiegel.png}
\caption{Rohdaten der $\chi'$-Messung}
\label{fig:Supra_X1roh}
\end{figure}

Die Rohdaten der $\chi'$-Messung und die vom Untergrund bereinigten Daten sind in Abbildung \ref{fig:Supra_X1roh} dargestellt.\\
Da für diese Messung statt einer Phase von $0^\circ$ eine von $180^\circ$ verwendet wurde, sind die Daten an der X-Achse gespiegelt. Außerdem entsteht in dieser Messung ein relativ großer Offset.\\
Für $T>T_C$ ist der Supraleiter dia- oder paramagnetisch, es gilt also $|\chi'| << 1$. Falls man annimmt, dass für diese Temperaturen $|\chi'| \approx 0$ gilt, so kann man den Offset korrigieren.
Die vom Untergund korrigierten, gespiegelten und vom Offset bereinigten Daten sind in Abbildung 
\subsection{Messung der Probe}

\section{Fazit}

\end{document}
