\documentclass[12pt,a4paper]{article}
\usepackage[utf8]{inputenc}
\usepackage[german]{babel}
\usepackage[T1]{fontenc}
\usepackage{amsmath}
\usepackage{amsfonts}
\usepackage{amssymb}
\usepackage{graphicx}
\usepackage{siunitx}
\usepackage{float}
\usepackage[left=2cm,right=2cm,top=2cm,bottom=2cm]{geometry}
\usepackage{hyperref}
\author{Gerald}


\begin{document}
\sisetup{separate-uncertainty = true}
	\setlength{\parindent}{0pt} 
	\begin{center}
		{\LARGE Versuchsprotokoll}\\
		\begin{large}
			zum Fortgeschrittenenpraktikum im Bachelorstudiengang Physik\\[0.4cm]
			an der RWTH Aachen\\
			II. Physikalisches Institut A\\[5.5cm]
			\Large\textbf{\textsl{Röntgenspektroskopie (T06)}}\\[5.5cm]
			\normalsize\textit{vorgelegt\\von}\\[0.4cm]
			\large{Moritz Berger (355244)\\Gerald Kolter (355005)}\\\textbf{Gruppe 30}\\[2cm]
			\large \textbf{Wintersemester 2017/18}
		\end{large}
	\end{center}
	\newpage
	
	\tableofcontents
	\newpage
	
\section{Hintergund}
Wenn man Materialien mit hochenergetischen Teilchen bestrahlt, so wechelwirken diese folgendermaßen mit den darin enthaltenden Atomen\footnote{Quelle:}:
\begin{enumerate}
\item Bei der Bewegung durch das Material werden die Teilchen abgebremst. Die dabei freiwerdende Energie äußert sich als ein kontinuirliches elektromagnetisches Spektrum, das Bremsstrahlung genannt wird. Die Intensität dieser Strahlung nimmt mit der Masse der abgebremsten Teilchen ab.
\item Die Teilchen geben ihre Energie an ein Elekron im Atom ab, was infolgedessen ionisiert wird. Dadruch wird eine Elektronposition frei und ein höherenergetisches Elektron im Atom kann auf diese Position fallen, wobei ein Röntgenquant abgestrahlt wird. Diesem Übergang kann eine diskrete Energie zugeordent werden. In der Realität äußert er sich wegen Energieunsicherheiten als Gau
\end{enumerate}
Der zweite Effekt hängt direkt von dem Atomaufbau ab, wodurch sich das dadurch ausgestrahlte Spektrum je nach Element unterscheidet. Dadurch kann man die Zusammensetzung eines Materials untersuchen, was in diesem Versuch ausgenutzt wird.
In diesem Versuch wird sowohl ein $\alpha$-Strahler, als auch eine Röntgenquelle als Quelle der hochenergetischen Teilchen benutzt.\\
Ziel ist es, beide Aufbauten mithilfe von mehreren Proben bekannter Zusammensetzung zu kalibrieren, um dann eine Reihe von unbekannten Proben mithilfe ihrer Röntgenpeaks zu untersuchen.
\section{Aufbau}
\subsubsection{$\alpha$-Quelle}
Der erste Aufbau besteht einem X-123 Spektrometer, welches aus einem Halbleiterdetekor der Kennung XR100CR und einem digitalen Pulsprozessor DP5 besteht, der die vom Detektor regristrierten Signale verstärkt und digitalisiert. Diese Signale werden an einen PC weitergeleitet, wo sie mit dem Programm und der Software ADMCA\footnote{http://amptek.com/products/dpp-mca-display-acquisition-software/} mithilfe eines Multi-Channel-Analyzers (MCA) nach ihrer Energie auf 496 Kanäle aufgeteilt werden.\\
Zur Kalibration dieser Channels gegen die Energie wird eine variable Röntgenstrahlquelle der Kennung 0317LA benutzt, welche aus einer $^{241}AM$-Quelle besteht, die $\alpha$-Strahlung auf ein durch eine Drehscheibe variables Target strahlt. Diese Quelle wird direkt vor den Detektor gestellt.\\
Zur Aufnahme von unbekannten Proben wird eine $^{241}AM$-Quelle der Kennummer Z3715 benutzt, welche in eine Bleibefestigung eingelassen wird.
\subsubsection{Röntgenquelle}
Bei den Messungen mit der Röntgenquelle wird ein X-123 Spektrometer der gleichen Bauart wie oben beschrieben verwendet. Die Röntgenquelle hat die Kennnummer BJ8288.
\section{Durchführung $\alpha$-Quelle}
\subsubsection{Kalibration}
Die zur Kalibration benutzte variable Röntgenstrahlquelle besitzt 6 Targets aus verschiedenen Materialien: Cu, Rb, Mo, Ag, Ba und Tb.
\section{Durchführung Röntgenquelle}
\section{Ergebnisse $\alpha$-Quelle}
\subsubsection{Kalibration}
\subsubsection{Auswertung unbekannter Proben}
\subsubsection{Energieauflösung}
\section{Ergebnisse Röntgenquelle}
\subsubsection{Kalibration}
\subsubsection{Auswertung unbekannter Proben}
\subsubsection{Energieauflösung}

\section{Fazit}
\section{Anhang}


\end{document}