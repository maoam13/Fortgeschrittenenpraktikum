\documentclass[12pt,a4paper]{article}
\usepackage[utf8]{inputenc}
\usepackage[german]{babel}
\usepackage[T1]{fontenc}
\usepackage{amsmath}
\usepackage{amsfonts}
\usepackage{amssymb}
\usepackage{graphicx}
\usepackage{siunitx}
\usepackage{float}
\usepackage[left=2cm,right=2cm,top=2cm,bottom=2cm]{geometry}
\author{Gerald}

\begin{document}
\sisetup{separate-uncertainty = true}
	\setlength{\parindent}{0pt} 
	\begin{center}
		{\LARGE Versuchsprotokoll}\\
		\begin{large}
			zum Fortgeschrittenenpraktikum im Bachelorstudiengang Physik\\[0.4cm]
			an der RWTH Aachen\\
			II. Physikalisches Institut A\\[5.5cm]
			\Large\textbf{\textsl{Gasdetektoren und Statistik (T07)}}\\[5.5cm]
			\normalsize\textit{vorgelegt\\von}\\[0.4cm]
			\large{Moritz Berger (355244)\\Gerald Kolter (355005)}\\(Gruppe 30)\\[2cm]
			\large \textbf{Wintersemester 2017/18}
		\end{large}
	\end{center}
	\newpage
	
	\tableofcontents
	\newpage
	
	
\section{Versuchsziel}
Das Ziel dieses Versuches ist in zwei Teile aufgeteilt.\\
Als erstes sollen die Eigenschaften eines Proportionalzählers und eines Geiger-Müller-Zählrohres untersucht werden. Dazu wird zum einen die Charakteristik, also die Abhängigkeit der Zählrate von der angelegten Spannung, untersucht, um den optimalen Bereich für die Betriebsspannung zu erhalten. Es wird außerdem die Abhängigkeit der Stärke eines Pulses des Proportionalzählers von der Spannung betrachtet und die Tot- und  Erhohlzeit des Geiger-Müller-Zählers bestimmt.\\
Im zweiten Versuchsteil sollen Messungen durchgeführt werden, die verschiedene vorhergesagte Verteilungen bestätigen. Dabei soll eine Gaussverteilung für eine hohe Anzahl an Messungen, eine Poissonverteilung für kleine Messergebnisse und eine Exponetialverteilung für den Abstand zwischen zwei gemessenen Zerfällen untersucht werden. 
\section{Aufbau}
\subsection{Geiger-Müller-Zählrohr}
\subsection{Proportionalzähler}

\section{Durchführung}
\subsection{Geiger-Müller-Zählrohr}
\subsubsection{Charakteristik}
Die Charakteristik wird mithilfe des $_{38}^{90}Sr$-Präparates gemessen. Dabei wird bei eingebrachtem Präparat die Spannung, ausgehend von der Minimaleinstellung von \SI{250}{V}, in \SI{10}{V} Schritten erhöht und jeweils über \SI{10}{s} die Anzahl der gemessenen Ereignisse erfasst.\\
Um die Einsatzspannung $U_E$ und die Geiger-Schwelle $U_G$ genauer bestimmen zu können, wurde im entsprechenden Bereich, in dem sich diese Werte befinden, in \SI{2}{V} Schritten statt in \SI{10}{V} Schritten gemessen.\\
\\Eine Schnellauswertung ergab $U_G = \SI{350}{V}$. Um sicher zu gehen, dass das Zählrohr für die folgenden Messungen im richtigen Spannungsbereich arbeitetet, wird die Spannung auf $U = U_G + 100V = \SI{450}{V}$ eingestellt.

\subsubsection{Untergrundrauschen}
Um eine mögliche Störung durch Untergrundrauschen zu beseitigen, wurde mit einer leeren Kammer insgesamt 5 mal über $\SI{1}{min}$ die Pulsrate gemessen.

\subsubsection{Totzeit}
Zur Totzeit des Geiger-Müller-Zählrohres sollen zwei verschiedene Messungen gemacht werden.
\paragraph{Stever-Diagramm}
Als erstes wird ein Stever-Diagramm erstellt, an dem man die Tod- und Erholungszeit ablesen kann. Dazu wird auf dem Oszilloskop ein unendliches Nachleuchten eingestellt. Der Trigger sollte dabei niedrig genug eingestellt sein, um möglichst alle Pulse zu erkennen. Durch die richtige Wahl der Zeitskala in der Größenordnung von einigen \SI{100}{\mu s} sollte die Einhüllender der Erholungszeit sichtbar werden. Anhand dieser wird nun händisch die Tot- und Erholungszeit 

\paragraph{Totzeitkorrektur}
Als zweites wird die Auflösungszeit mithilfe einer Totzeitstufe bestimmt. Mithilfe dieser kann man eine Totzeitkorrektur der gemessenen Zählraten durchführen. Dabei ist die wirkliche Zählrate N in Abhängigkeit von der gemessenen Rate n und der Auflösungszeit $\tau$ gegeben durch 
\begin{equation}
\label{Totzeitkorrektur}
N = \dfrac{n}{1-n\tau}
\end{equation}
Nun werden für zwei verschiedene künstlich angelegte Totzeiten $\tau_{2ms}$ und $\tau_{1\mu s}$ die Zählrate bestimmt. Dazu werden zuerst die Frequenzen der Totzeitstufe überprüft, um sicher zu stellen, dass diese richtig kalibriert ist. Dafür wurde über 1s die Anzahl der durchgelassenen Pulse gemessen.\\
Danach werden die beiden Zählraten bestimmt. Dazu wird für beide Zeiten 1min lang die Zählrate gemessen. Da N für beide Totzeiten gleich sein sollte und $\tau_{1\mu s} << \tau$ gilt, kann man durch Gleichsetzten von Gleichung \ref{Totzeitkorrektur} für beide Zeiten die Auflösungszeit bestimmen:
\begin{equation}
\tau_{1\mu s} \approx \tau = \dfrac{1}{n_{1\mu s}} - \dfrac{1}{n_{2ms}} + \tau_{2ms}
\label{Auslosungszeit}
\end{equation}

Die Messung wurde zuerst mit der Probe auf der mittleren Schiene durchgeführt, und dann mit der Probe auf der 2. Schiene von oben wiederholt, um eine möglichst hohe Peakrate zu messen.
\subsection{Proportionalzähler}

\subsubsection{Charakteristik}
Beim Proportionalzähler wurde ebenfalls die Charakteristik vermessen, und zwar einmal Anhand der $_{95}^{241}AM$-Probe und einmal anhand der $_6^{14}C$-Probe. Dazu wurde in \SI{20}{V} oder (aus Zeitgründen) in \SI{40}{V} Schritten, wieder ausgehend von \SI{250}{V}, jeweils über \SI{10}{s} die Pulsrate aufgenommen. Bei \SI{1200}{V} musste die Hochspannungsquelle auf 2 fache Verstärkung umgestellt werden. Es kann sein, dass die in Wirklichkeit angelegte Spannung für die beiden Verstärkungsmodi nicht gleich ist, was sich auf die Messdaten auswirken kann.\\
Da beim Proportionalzähler die Pulsstärke abhängig von der Energie der Teilchen ist, wurde zusätzlich noch die Pulshöhe mithilfe des Oszilloskopes für alle Spannungswerte bestimmt.\\
Außerdem wurde ebenfalls eine Untergrundmessung mit leerem Probenraum für den vollständigen untersuchten Spannungsbereich durchgeführt.

\subsection{Statistikmessung}
Die Statistikmessungen wurden alle mit dem Geiger-Müller-Zählrohr aufgenommen.
\subsubsection{Poissonverteilung}
Die Zerfälle sind poissonverteilt. Um eine Poissonverteilung auch in den Messdaten beobachten zu können, muss die Messzeit klein gehalten werden (T = 0.3s), da die Verteilung ansonsten in eine Gaußverteilung übergeht. Die Poissonverteilung ist nur für diskrete Werte definiert und hat die Form:
\begin{equation}
P(n) = \dfrac{\mu ^n}{n!} \cdot e^{-\mu}
\end{equation}
\subsubsection{Gaußverteilung}
Da die Zerfälle poissonverteilt sind, muss für die Messung einer Gaußverteilung die Messzeit um Faktor 10 erhöht werden (T = 3s). Die Gaußverteilung hat die Form:
\begin{equation}
P(x) = \dfrac{1}{\sqrt{2 \pi} \sigma} \cdot e^{\frac{(x - \mu)^2}{2 \cdot \sigma ^2}}
\end{equation}
\subsubsection{Exponentialverteilung}
Die Abstände zwischen zwei Zerfällen sind exponentialverteilt. Zur Messung werden mit dem Oszilloskop Bilder gemacht, auf denen mehrere Zerfälle zu sehen sind. Die Exponentialverteilung hat die Form:
\begin{equation}
P(\Delta t) = A \cdot e^{- A \Delta t}
\end{equation}

\section{Ergebnisse}
\subsection{Geiger-Müller-Zählrohr}

\subsubsection{Totzeit mithilfe des Stever-Diagramms}

\begin{figure}
\centering
\includegraphics[scale=0.49]{Bilder/Stever/Stever1_1.PNG}
\includegraphics[scale=0.49]{Bilder/Stever/Stever1_2.PNG}
\caption{...}
\label{fig:Stever}
\end{figure}

\begin{figure}
\centering
\includegraphics[scale=0.49]{Bilder/Stever/Stever2_1.PNG}
\includegraphics[scale=0.49]{Bilder/Stever/Stever2_2.PNG}
\caption{...}
\label{fig:Stever2}
\end{figure}


In Abbildung \ref{fig:Stever} und \ref{fig:Stever2} sind die Oszilloskopbilder für zwei unterschiedliche Zeitskalen mit jeweils abgelesenen Tot- und Erholungszeiten dargestellt. Leider konnten die Daten nicht aus dem Oszilloskop extrahiert werden, weswegen die Auswertung vollständig visuell gemacht werden muss.\\
Die Totzeit wird anhand der linken Grenze der Einhüllenden bestimmt. Da durch die eingestellte Schwelle des Verstärkers zu kleine Pulse nicht registriert werden, ist die echte Totzeit etwas kleiner als die abgelesene Grenze.\\
Die Erholungszeit wird bestimmt, indem der Punkt abgelesen wird, ab dem die Peakhöhe einen konstanten Wert annimmt.
Sie kann man nicht so genau bestimmt werden, da durch die große statistische Schwankung der Peaks nicht genau erkennbar ist, an welchem Punkt diese Bedingung erreicht ist.\\
Die abgelesenen Ergebnisse sind in Tabelle \ref{teb:Stever} aufgelistet. Die Fehler wurden durch die Ableseungenauigkeit abgeschätzt.

\begin{table}
\centering
\begin{tabular}{|c|c|c|}
\hline
Zeitskala & Totzeit & Erholungszeit\\
\hline 
250 & $220\pm 20$ & $520\pm 50$ \\
\hline 
250 & $220\pm 10$ & $450\pm 30$ \\
\hline
\end{tabular}
\label{teb:Stever}
\end{table}

Eine gewichtete Mittelung ergibt (mit äußerem Fehler):

\begin{equation}
\boxed{T_D = \SI{220 \pm 9}{\mu s}}
\end{equation}

\begin{equation}
\boxed{T_R = \SI{469 \pm 31}{\mu s}}
\end{equation}

\subsubsection{Totzeit mithilfe der Totzeitstufe}

Vor der eigentlichen Messung wurde die Kalibration der Totzeitstufe durchgeführt. Die Ergebnisse sind in Tabelle \ref{tab:Totzeitkalibration} aufgelistet. Die Messung wurde mehrfach wiederholt, um sicher zu stellen, dass das Ergebnis exakt ist. Es ergab sich mehrfach derselbe Wert, weswegen das Ergebnis auf $1\mu s$ exakt angenommen werden kann. Relevant für die weitere Auswertung ist nur die $2ms$ Stufe, die auf $1.908ms$ kalibriert wurde.

\begin{table}
\centering
\begin{tabular}{|c|c||c|}
\hline
Einstellung & $1\mu$ s & $2ms$\\
\hline 
Pulsrate & $1000003$ & $524$ \\
\hline 
Totzeit & $1\mu s$ & $1.908ms$ \\
\hline
\end{tabular}
\label{tab:Totzeitkalibration}
\end{table}

\begin{table}
\centering
\begin{tabular}{|c|c||c|}
\hline
Messzeit[s] & 60 & 60\\
\hline 
$n_{1\mu s}$ & $10603 \pm 103$ & $5259\pm 73$ \\
\hline 
$n_{2ms}$ & $8248\pm 91$ & $4599\pm 68$ \\
\hline 
$\tau [\mu s]$ & $324\pm 97 $ & $1.877\pm 0.004$ \\
\hline
\end{tabular}
\label{tab:Totzeitstufe}
\end{table}

In Tabelle \ref{tab:Totzeitstufe} sind die Ergebnisse der beiden Messungen zur Totzeitstufe aufgelistet. Der Fehler wurde mit $\sqrt{n}$ angenommen.\\
Daraus wurde mithilfe von Gleichung \ref{Totzeitkorrektur} die Totzeit bestimmt, welche ebenfalls in der Tabelle aufgelistet ist. Der Fehler wurde folgendermaßen errechnet:

\begin{equation}
\sigma_{\tau} = \sqrt{\left(\dfrac{\sqrt{n_{1\mu s}}}{n_{1\mu s}^{2}}\right)^2 + \left(\dfrac{\sqrt{n_{2ms}}}{n_{2ms}^{2}}\right)^2 }
\end{equation}

Eine gewichtete Mittelung der $\tau$ ergibt:
\begin{equation}
\boxed{\tau = \SI{1.881\pm 0.002}{ms}}
\end{equation}


\subsubsection{Charakteristik}
\label{GMchar}
\begin{figure}
\centering
\includegraphics[scale=1]{Bilder/GMlinear.PNG}
\includegraphics[scale=1]{Bilder/GMlog.PNG}
\caption{...}
\label{fig:GMCharakteristik}
\end{figure}
Die Ergebnisse der Charakteristikmessung sind in Abbildung \ref{fig:GMCharakteristik} dargestellt. Diese wurden bereits mithlife von Gleichung \ref{Auslosungszeit} von der Totzeit bereinigt.
Außerdem wurde mithilfe der Rauschmessung der Offset des Untergrundrauschens abgezogen.\\
Mit zunehmender Spannung steigt die Messrate exponentiell, da immer mehr Ereignisse mit kleineren Energien registriert werden. Daher Müssen die Daten noch logarythmisiert werden, um das Plateau erkennen zu können.\\
Der relativ große Fehler auf die Totzeit ist dabei von systematischer Natur, er verschiebt also die Werte in die gleiche Richtung. Da das Plateau in der logarythmischen Sklaierung eben ist, verschiebt er die Werte um nahezu den gleichen Faktor, womit sich die Position der Einsatz- und Geiger-Spannung nicht nennenswert ändert. Der Fehler kann also für die Bestimmung dieser vernachlässigt werden.

\paragraph{Bestimmung der Geiger-Spannung}
\begin{figure}
\centering
\includegraphics[scale=1]{Bilder/GMGeradenfit.PNG}
\caption{...}
\label{fig:GMGeradenfit}
\end{figure}

\begin{figure}
\centering
\includegraphics[scale=1]{Bilder/GMKurvenfit.PNG}
\caption{...}
\label{fig:GMKurvenfit}
\end{figure}

Anhand der logarythmisierten Daten wird nun die Geiger-Spannung $U_G$ bestimmt. Dazu  wurde als erstes abgeschätzt, dass das Plateau spätestens bei $\SI{360}{V}$ erreicht wird und ein Geradenfit an die Daten überhalb dieser Spannung angepasst. Dieser ist in Abbildung \ref{fig:GMGeradenfit} zu sehen.\\
Um nun den Anfang des Plateaus bestimmen zu können, wird außerdem noch ein Fit der Form
\begin{equation*}
y = \dfrac{a}{x+d} + b\cdot x + c
\end{equation*}
an die Daten angefitted. Es wurde hier eine $\dfrac{1}{x}$ Funktion gewählt, da es wegen der wenigen Datenpunkte, die groß von der Plateaugeraden abweichen, sehr schwierig gewesen ist eine Funktion an die Daten anzupassen und diese Funktion die einzige ist, bei der der Fit geglückt ist und die Umgebung der Geigerfrequenz gut beschreibt. Das Ergebnis ist in Abbildung \ref{fig:GMKurvenfit} dargestellt.\\
$U_G$ wird bestimmt, indem die Spannung gesucht wird, ab der die beiden Anpassungen innerhalb ihrer Unsicherheiten überseinstimmen. Es muss aber beachtet werden, dass die Anpassungen korrelieren, da sie aus demselben Datensatz stammen. Der Einfachheit halber wird angenommen, dass der Fehler des Linearen Fits vollständig mit dem des$\dfrac{1}{x}$-Fit korreliert, der denselben linearen Term enthält. Ist der tatsächliche lineare Fit beispielsweise leicht nach oben verschoben, so ist mit dieser Annahme der echte $\dfrac{1}{x}$-Fit ebenfalls um den gleichen Faktor nach oben verschoben. Somit ergibt sich der Abstandsfehler zwischen den beiden Fits als quadratische Differenz der beiden Unsicherheiten.\\
Den Fehler auf $U_G$ erhält man aus der Verschiebemethode. Die beiden Fits werden also um ihre Fehler nach oben oder unten verschoben und ein neuer Wert für $U_G$ mit der oben beschriebenen Methode bestimmt.\\
Mit dieser Methode ergibt sich:
\begin{equation*}
\boxed{U_G = (346_{-5}^{+7})\si{V}}
\end{equation*}
\paragraph{Bestimmung der Einsatz-Spannung}
Im Gegensatz zu $U_G$ lässt sich die Einsatz-Spannung $U_E$ direkt aus den Daten ablesen, da diese Spannung sich genau an der Stelle befindet, an der zum ersten mal ein Puls gemessen wird. Dies war der Fall bei
\begin{equation*}
\boxed{U_G = \SI{316\pm 1}{V}}
\end{equation*}
Der Fehler ergibt sich durch die Einstellgenauigkeit der Hochspannung.

\subsubsection{Zusammenfassung}



\subsection{Proportionalzähler}

\subsubsection{Pulsraten}

\begin{figure}
\centering
\includegraphics[scale=0.8]{Bilder/Prop/Am_lin.PNG}
\includegraphics[scale=0.8]{Bilder/Prop/Am_log.PNG}
\caption{AM}
\label{fig:amlin}
\end{figure}
\begin{figure}
\centering
\includegraphics[scale=0.8]{Bilder/Prop/C_lin.PNG}
\includegraphics[scale=0.8]{Bilder/Prop/C_log.PNG}
\caption{CH}
\label{fig:clin}
\end{figure}

Die Pulsraten für das Americium-Präparates sind in Abbildung  \ref{fig:amlin} dargestellt und die des Kohlenstoff-Präparates in Abbildung \ref{fig:clin}. Dabei wurden die Daten aus dem gleichen Grund, wie beim Geiger-Müller-Zählrohr, logarythmisiert (vgl. Kapitel \ref{GMchar}).\\
\paragraph{C-Präparat}
$_6^{14} C$ ist ein $\beta$-Strahler. Aus diesem Grund ist wie erwartet nur das Plateau in den Daten zu sehen, in das sowohl $\alpha$ als auch $\beta$ Strahlung mit einfließen.\\
Um die Breite des Plateaus abschätzen zu können wurde ein Linearer fit an die logarythmisierten Werte angepasst und abgeschätzt, welche gemessenen Spannungswerte  mit dieser Anpassung übereinstimmt. Die Anpassung ist in Abbildung \ref{fig:CPlateau} dargestellt. Offensichtlich beginnt das Plateau bei $\SI{1570\pm 12}{V}$ und geht bis $\SI{1930\pm 12}{V}$. Aus dem Fit ergibt sich außerdem eine Steigung von $a = (5.7\pm0.3)\cdot 10^{-4}$. Dies entspricht einer Zunahme der Steigung der Zählrate von 0.5 über das ganze Plateau.
\begin{figure}
\centering
\includegraphics[scale=0.8]{Bilder/Prop/C_plateau.PNG}
\caption{...}
\label{fig:CPlateau}
\end{figure}\\
\\
Erst ab einer Spannung von $\SI{1130\pm 12}{V}$ werden überhaupt nennenswerte Pulsraten gemessen. Anscheinend wird ab dieser Spannung also $\beta$-Strahlung gemessen, dass $\alpha$-Plateau hört also vermutlich bei dieser Spannung auf.

\paragraph{AM-Präparat}
Armericium ist sowohl ein $\alpha$-, als auch ein $\beta$-Strahler. Theoretisch sollten also beide Plateaus in der Messung zur Pulsrate sichtbar sein. Im den Daten ist aber nur ein Plateau sichtbar. Dieses wird genau so wie das Plateau des C-Präparates abgeschätzt, was in Abbildung \ref{fig:AMPlateau} zu sehen ist.

\begin{figure}
\centering
\includegraphics[scale=0.8]{Bilder/Prop/Am_plateau.PNG}
\caption{...}
\label{fig:AMPlateau}
\end{figure}

Das Plateau befindet sich in einem Spannungsbereich von $\SI{760\pm 6}{V}$ bis $\SI{1180\pm 6}{V}$ und hat eine Steigung von $a = (8.6\pm1.4)\cdot 10^{-5}$. Dies entspricht einer Zunahme der Zählratensteigung von 0.1 im Plateaubereich.\\ Das Plateauende ist mit dem Ergebnis der C-Proben Auswertung verträglich, bei dem die Zählung von $\beta$-Strahlung bei c.a 1130V einsetzt. Im Bereich bis 1180V ist die Zunahme der Zählrate durch diese Strahlung so klein, dass man sie nicht aus der Messstreuung herausfiltern kann.\\
\\
Das zweite Plateau, welches nach den Ergenissen von der C-Probe bei c.a 1570V anfangen sollte, fehlt in diesen Messdaten. Stattdessen fängt die Zählrate bei hohen Spannungen an zu sinken. \\
Dieser Effekt ist unerwünscht und kann viele Ursachen haben.\\
\\
Die Einsatzspannung ist hier wesentlich kleiner als bei der C-Probe und liegt außerhalb der unseres Messbereiches. Dies liegt daran, dass die durch die Americium-Zerfallsreihe entstehende $\alpha$-Strahlung eine Energie besitzt, die 
um ungefähr ein zehnfaches größer ist als die $\beta$-Strahlung der C-Probe. Dadurch finden schon bei niedrigen Spannungen genug Inonisationsprozesse statt, um einen Messpuls auszulösen.\\




\newpage


\subsection{Statistikmessung}

\subsubsection{Poissonverteilung}
\begin{figure}
\centering
\includegraphics[scale=0.8]{Bilder/poisson.PNG}
\caption{Darstellung der gemessenen Poissonverteilung mit den Erwartungen und der Anpassung.}
\label{fig:Poisson}
\end{figure}

\begin{table}
\centering
\begin{tabular}{|c|c|c|c|}
\hline 
Art & Poisson Parameter & $\chi ^2$ /ndof & P($\chi ^2$) \\ 
\hline 
Mittelwertbestimmung & 1,0900 & 45,850 & 0\% \\ 
\hline 
Bestimmung der Standardabweichung & 0,9918 & 45,827 & 0\% \\ 
\hline 
$\chi ^2$ Anpassung & 0,9332 $\pm$ 0,3689 & 11,125 & 0\% \\ 
\hline 
\end{tabular} 
\caption{Poisson Parameter und $\chi ^2$ aus den verschiedenen Bestimmungsmethoden der Poissonverteilung. In der letzten Spalte ist eine geschätzte Wahrscheinlichkeit für den Wert des $\chi ^2$ angegeben.}
\label{tab:Poisson}
\end{table}

Für die Poissonverteilung wird die Anzahl der Pulse im Geiger-Müller-Zählrohr für ein festgelegtes Messintervall wiederholt gemessen. Die auftretenden Anzahlen werden dann in Intervalle mit Breite 1 zusammengefasst und aufgetragen. Das Ergebnis zeigt das Histogramm in Abbildung \ref{fig:Poisson} zusammen mit der $\chi ^2$ Anpassung (in rot) und den Erwartungen die sich aus der Bestimmung des Mittelwertes (grün) und der Standardabweichung (gelb) ergeben.\\
Für die Anpassung wurde auf jeden Wert ein Fehler gemäß der Poissonverteilung angenommen:
\begin{equation*}
\sigma _{n} = \sqrt{n}
\end{equation*}
Für die Anpassung einer Gaußverteilung wurden die Werte normiert, daher pflanzt sich der Fehler fort zu:
\begin{equation*}
n_{i, norm} = \dfrac{n_i}{\sum _j n_j} =: \dfrac{n_i}{N}
\end{equation*}
\begin{equation*}
\sigma _{i, norm} = \sqrt{ \sum _j \left( \sigma_j \cdot \left( \delta_{ij} \cdot \left( \frac{1}{N} - \frac{n_i}{N} \right) + (1 - \delta_{ij}) \cdot \frac{n_i}{N^2} \right) \right)^2 }
\end{equation*}
AB HIER POISSON ÜBERARBEITEN!! \\
Tabelle \ref{tab:Poisson} zeigt die Ergebnisse aus den verschiedenen Methoden zur Bestimmung des Poisson Parameters und des $\chi ^2$. Leicht lässt sich ersehen, dass die Werte für den Poisson Parameter nah beieinander liegen, das $\chi ^2$ aus der Anpassung jedoch deutlich kleiner ist als die aus der Berechnung nach Bestimmung des Parameters. Insgesamt ist das $\chi ^2$ jedoch deutlich zu groß. 


\subsubsection{Gaußverteilung}
\begin{figure}
\centering
\includegraphics[scale=0.8]{Bilder/gauss.PNG}
\caption{Darstellung der gemessenen Gaußverteilung mit der Erwartung und der Anpassung.}
\label{fig:gauss}
\end{figure}

\begin{table}
\centering
\begin{tabular}{|c|c|c|c|c|}
\hline 
Art & Mittelwert & Standardabweichung & $\chi ^2$ /ndof & P($\chi ^2$) \\ 
\hline 
Bestimmung von $\sigma$ und $\mu$ & 27,375 & 4,611 & 0,688 & 30\% \\ 
\hline 
$\chi ^2$ Anpassung & 27,286 $\pm$ 0,307 & 4,578 $\pm$ 0,260 & 0,705 & 30\% \\ 
\hline 
\end{tabular} 
\caption{Mittelwert, Standardabweichung und $\chi ^2$ aus den verschiedenen Bestimmungsmethoden der Gaußverteilung. In der letzten Spalte ist eine geschätzte Wahrscheinlichkeit für den Wert des $\chi ^2$ angegeben.}
\label{tab:Gauss}
\end{table}

Das Verfahren bei der Gaußverteilung ist identisch zu dem Verfahren bei der Poissonverteilung. Es wurde auch die gleiche Fehlerabschätzung verwendet. Abbildung \ref{fig:gauss} zeigt das Ergebnis. Hier ist in rot die $\chi ^2$ Anpassung und die Erwartung aus Mittelwert und Standardabweichung in grün. In diesem Fall gibt es nur eine Erwartung, weil für die Bestimmung der Gaußverteilung Mittelwert und Standardabweichung notwendig sind.\\
Tabelle \ref{tab:Gauss} zeigt die Werte für Mittelwert, Standardabweichung und $\chi ^2$, die sich aus dem Fit und aus der Erwartung durch Bestimmung von Standardabweichung und Mittelwert ergibt. Die Werte für Mittelwert und Standardabweichung stimmen sehr gut überein und die $\chi ^2$ liegen ebenfalls sehr nah beieinander. Die Wahrscheinlichkeit für ein $\chi ^2$ pro Freiheitsgrad von 0,7 liegt ungefähr bei 30\%, daher kann von einer guten Messung gesprochen werden.


\subsubsection{Exponentialverteilung}
\begin{figure}
\centering
\includegraphics[scale=0.8]{Bilder/Peakabstaende.PNG}
\caption{Darstellung der gemessenen Peakabstände mit den Erwartungen und der Anpassung.}
\label{fig:exponential}
\end{figure}

\begin{table}
\centering
\begin{tabular}{|c|c|c|c|}
\hline 
Art & Exponentialparameter & $\chi ^2$ /ndof & P($\chi ^2$) \\ 
\hline 
Bestimmung von $\mu$ & 0,1143 & 11,307 & 0\% \\ 
\hline 
Bestimmung von $\sigma$ & 0,1181 & 11,302 & 0\% \\ 
\hline 
$\chi ^2$ Anpassung & 0,3362 $\pm$ 0,053 & 1,129 & 22\% \\ 
\hline 
\end{tabular} 
\caption{Exponential Parameter und $\chi ^2$ aus den verschiedenen Bestimmungsmethoden der Exponentialverteilung. In der letzten Spalte ist eine geschätzte Wahrscheinlichkeit für den Wert des $\chi ^2$ angegeben.}
\label{tab:Exponential}
\end{table}

Für die Auswertung der Exponentialverteilung werden die gemessenen Zeitabstände zunächst in sinnvolle Intervalle zusammengefasst. Die Anzahl der gemessenen $\Delta t$, die in diesem Intervall liegen werden auf die x-Achse aufgetragen. Die Anzahl der Intervalle mit der gleichen Anzahl der gemessenen $\Delta t$ wird auf die y-Achse aufgetragen. Dabei wird wieder normiert, daher kann wieder die gleiche Fehlerabschätzung wie bei der Poissonverteilung verwendet werden. Abbildung \ref{fig:exponential} zeigt das Ergebnis. Hier gibt es wieder zwei Erwartungen aus Bestimmung des Mittelwertes (grün) und Bestimmung der Standardabweichung (gelb). Die $\chi ^2$ Anpassung ist wieder in rot eingezeichnet.\\
Tabelle \ref{tab:Exponential} zeigt die Ergebnisse für den Exponentialparameter und das $\chi ^2$ pro Freiheitsgrad für die verschiedenen Bestimmungsmethoden. Auffällig ist, dass die Exponentialparameter aus den Bestimmungen von Mittelwert und Standardabweichung nah beieinander liegen, der Wert aus der $\chi ^2$ Anpassung jedoch deutlich von diesen abweicht. Das $\chi ^2$ ist bei der Anpassung deutlich besser, als bei den Bestimmungen von Mittelwert und Standardabweichung, sodass davon ausgegangen werden kann, dass der Wert aus der Anpassung näher am echten Wert liegt.

\section{Zusammenfassung}




	
\end{document}