% ****** Start of file apssamp.tex ******
%
%   This file is part of the APS files in the REVTeX 4.1 distribution.
%   Version 4.1r of REVTeX, August 2010
%
%   Copyright (c) 2009, 2010 The American Physical Society.
%
%   See the REVTeX 4 README file for restrictions and more information.
%
% TeX'ing this file requires that you have AMS-LaTeX 2.0 installed
% as well as the rest of the prerequisites for REVTeX 4.1
%
% See the REVTeX 4 README file
% It also requires running BibTeX. The commands are as follows:
%
%  1)  latex apssamp.tex
%  2)  bibtex apssamp
%  3)  latex apssamp.tex
%  4)  latex apssamp.tex
%
\documentclass[%
 reprint,
%superscriptaddress,
%groupedaddress,
%unsortedaddress,
%runinaddress,
%frontmatterverbose,
%preprint,
%showpacs,preprintnumbers,
%nofootinbib,
%nobibnotes,
%bibnotes,
amsmath,amssymb,
%aps,
pra,
%prb,
%rmp,
%prstab,
%prstper,
%floatfix,
]{revtex4-1}

\usepackage{siunitx}
\usepackage{graphicx}% Include figure files
\usepackage{dcolumn}% Align table columns on decimal point
\usepackage{bm}% bold math
%\usepackage{hyperref}% add hypertext capabilities
%\usepackage[mathlines]{lineno}% Enable numbering of text and display math
%\linenumbers\relax % Commence numbering lines

%\usepackage[showframe,%Uncomment any one of the following lines to test
%%scale=0.7, marginratio={1:1, 2:3}, ignoreall,% default settings
%%text={7in,10in},centering,
%%margin=1.5in,
%%total={6.5in,8.75in}, top=1.2in, left=0.9in, includefoot,
%%height=10in,a5paper,hmargin={3cm,0.8in},
%]{geometry}

\begin{document}

\preprint{APS/123-QED}

\title{Fabrication and characterization of a Pseudo-MOSFET}% Force line breaks with \\

\author{Moritz Berger}
 \altaffiliation[]{RWTH Aachen University, Germany}%Lines break automatically or can be forced with \\
 \email{moritz.berger@rwth-aachen.de}
 \author{Gerald Kolter}
 \altaffiliation[]{RWTH Aachen University, Germany}%Lines break automatically or can be forced with \\
 \email{gerald.kolter@rwth-aachen.de}

\date{\today}% It is always \today, today,
             %  but any date may be explicitly specified

\begin{abstract}
bla
\end{abstract}

\maketitle

\section{Introduction}
In the following the fabrication of a Pseudo-MOSFET ("metallic-oxide-semiconductor-field-effect-transistor") will be described. The next step is a more detailed description of optical lithographie and reactive ion etching. Afterwards an analysis and a discussion of the characterization of the fabricated Pseudo-MOSFET will be given.

\section{Fabrication}

\begin{table}
\centering
\begin{tabular}{|c|c|}
\hline 
Fabrication technology & "UNIBOND" \\ 
\hline 
top Si thickness & \SI{85}{nm} \\ 
\hline 
buried oxide thickness & \SI{145}{nm} \\ 
\hline 
doping type & p-type (Boron) \\ 
\hline 
doping concentration & $1 \times 10^{15}$ \si{\per\cubic\centi\meter} \\ 
\hline 
crystal orientation & (100) \\ 
\hline 
\end{tabular} 
\caption{Specification of the SOI wafer.}
\label{tab:Spec_SOI}
\end{table}

The basis for the MOSFET are SOI (Semiconductor On Insulator) samples, the specifications are listed in table \ref{tab:Spec_SOI}. Si is used as semiconductor and SiO$_2$ as insulator. In a first step an mask for the following etching was defined lithographically. The mesa structuring is done by reactive ion etching with a gas mixture of SF$_6$/O$_2$. For removing the resist an oxygen plasma is used. At next the samples are RCA-cleaned ending with a dip in hydroflourid acid to get a bare silicon surface. The aluminum is deposited in a vacuum. The aluminum layer is structured with optical lithographie and an aluminum etching as which phosphoric acid, nitric acid, acetic acid and water in a volume ratio of 16:1:1:2 is used. 

\section{Theory}
\subsection{Optical Lithographie}

\subsection{Reactive Ion Etching}

\section{Data Analysis}

\section{Results and Discussion}

\section{Conclusion}

\bibliography{MOSFET}% Produces the bibliography via BibTeX.

\end{document}

\begin{thebibliography}{x}
   \bibitem[Esselborn-Krumbiegel, H., 2008]{bmbf} Von der Idee zum Text. Eine Anleitung zum wissenschaftlichen Schreiben. Paderborn: Verlag Ferdinand Sch�ningh GmbH \& Co. KG.

   \bibitem[Franck, N., 2004]{bmbf} Handbuch Wissenschafliches Arbeiten. Frankfurt am Main: Fischer Taschenbuch Verlag.
Karmasin, M., \& Ribing, R. (2011). Die Gestaltung wissenschaftlicher Arbeiten. Wien: Facultas Verlags- und Buchhandels AG.

   \bibitem[Lengauer, H., \& Wimmer, A., 2006]{bmbf} http://www.uni-klu.ac.at. Abgerufen am 30. 9. 2013 von Definition Plagiat: http://www.uni-klu.ac.at/main/inhalt/3054.htm

   \bibitem[Prei�ner, A., 2012]{bmbf} Wissenschaftliches Arbeiten: Internet nutzen, Text erstellen, �berblick behalten. M�nchen: Oldenbourg.

\end{thebibliography}
